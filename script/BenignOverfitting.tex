\documentclass[a4paper,12pt]{scrartcl}

% font packages
\usepackage[utf8]{inputenc}
\usepackage[T1]{fontenc}

% packages for mathematical type setting
\usepackage{amsmath}
\usepackage{amssymb}
\usepackage{amsthm}

% packge for norm command
\usepackage{physics}

% references
\usepackage{cleveref}

\author{\normalsize Linus Böhm, Jurek Rostalsky}
\title{Benign Overfitting}
\date{}

% formatting
\setlength{\parindent}{0pt}
\pagestyle{empty}

% definition
\theoremstyle{definition}
\newtheorem{definition}{Definition}[section]


\begin{document}
\maketitle
\section{Basic definitions} \label{sec:basic_definitions}
Let \(x \in \mathbb{H}, y \in \mathbb{R}\) with zero mean. Where \(\mathbb{H}\) is a Hilbert space.

\begin{definition} [covariance matrix]
	\label{def:covarianvce_matrix}
	\begin{equation}
		\Sigma = \mathbb{E}\left(\left(x - \mathbb{E}(x)\right)\left(x - \mathbb{E}(x)\right)^T\right) =  \mathbb{E}(xx^T)
	\end{equation} 
\end{definition}

\begin{definition} [linear regression]
	\label{def:linear_regression}
	The problem of finding a parameter vector \(\theta^\ast \in \mathbb{H}\) with
	\begin{equation}
		\theta^\ast = arg \min\limits_\theta \mathbb{E}\left((y - x^T \theta)^2\right)
	\end{equation}
	is called \textbf{linear regression}.
\end{definition}

Let \(\left((x_1, y_1), ..., (x_n, y_n)\right) \in (\mathbb{H} \times \mathbb{R})^n\) a list of \(n\) sampled data points. Now we define the matrix \(X = \big(x_1 \, x_2 \, ... \, x_n\big)\) and the vector \(y = (y_1 \, y_2 \, ... \, y_n)^T\). If there is a \(\theta \in \mathbb{H}\) with \(y - X^T \theta = 0\) that \(\theta\) is a minimum of the linear regression problem sind the expectation of a square is non negative. Usually such a \(\theta\) isn't unique, so we are interested in the minimum norm \(\theta\) with that property.

\begin{definition}[minimum norm estimator]
	For given samples \(X \in \mathbb{H}^n, y \in \mathbb{R}^n\). The \textbf{minimum norm estimator} \(\theta\) is the solution of the QQP:
		\begin{align*}
			\label{eq:QQP}
			\hat{\theta} = arg \min\limits_{\theta} &\norm{\theta}^2 & \text{subject to: } \norm{X^T \theta - y}^2 = \min\limits_\beta \norm{X^T \beta - y}^2 && \text{(QQP)}\\
		\end{align*}		
\end{definition}

The minimum norm estimator can be obtained by solving the normal equation:
\begin{equation}
\label{eq:normal_equation}
	XX^T \theta = X y,
\end{equation}
which can be done by numerical stable with QR-decomposition.


\end{document}