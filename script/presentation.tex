\documentclass{beamer}
%
% Choose how your presentation looks.
%
% For more themes, color themes and font themes, see:
% http://deic.uab.es/~iblanes/beamer_gallery/index_by_theme.html
%
\mode<presentation>
{
  \usetheme{JuanLesPins}      % or try Darmstadt, Madrid, Warsaw, ...
  \usecolortheme{default} % or try albatross, beaver, crane, ...
  \usefonttheme{default}  % or try serif, structurebold, ...
  \setbeamertemplate{navigation symbols}{}
  \setbeamertemplate{caption}[numbered]
} 

\usepackage[ngerman]{babel}
\usepackage[utf8x]{inputenc}
\usepackage{amsmath}
\usepackage{graphicx}

% norm
\usepackage{physics}

\title{Benign Overfitting}
\date{16. December 2021}
\author{Linus Boehm, Jurek Rostalsky}


\begin{document}
\maketitle

\begin{frame}{MNIST}
\begin{block} {MNIST}
\begin{itemize}
	\item very commonly used benchmark problem in machine learning
	\item goal: recognize hand written digit from an \(28 \cdot 28\) pixel image
	\item contains 60000 training images/ labels, 10000 test images/ labels
	\item standard ML approches perform very well
\end{itemize}
% put some images on that page

\end{block}
\end{frame}

\begin{frame} {least norm solution}
\begin{block}{normal euation}
problem: \(\min \frac{1}{2} x^Tx \text{ s.t. } Ax = b\) A with full row rank\\
\begin{align*}
	L(x,\lambda) &= \frac{1}{2} x^Tx - \lambda^T (Ax -b)\\
	\nabla_x L(x,\lambda) &= x - A^T \lambda \stackrel{!}{=} 0\\
	x & = A^T \lambda\\
	AA^T \lambda &= b \Leftrightarrow \lambda = (AA^T)^{-1} b\\
	x &= A^T(AA^T)^{-1} b \Leftrightarrow \\
\end{align*}
\end{block}
\end{frame}

\begin{frame}{least norm solution}
\begin{block}{QR decomposition}
\begin{align*}
	A^TAx &= A^T b & \text{normal equation}\\
	R^TQ^TQR^Tx &= R^TQ^Tb\\
	R^TRx & = R^TQ^Tb\\
	Rx &= Q^T b\\
\end{align*}

\end{block}
\end{frame}
\end{document}